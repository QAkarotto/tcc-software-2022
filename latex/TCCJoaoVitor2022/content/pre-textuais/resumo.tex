% RESUMO--------------------------------------------------------------------------------

\begin{resumo}[RESUMO]
\begin{SingleSpacing}


Esta monografia apresenta uma proposta de um método para implementação de testes automatizados em projetos de software, utilizando como estudo de caso uma aplicação \emph{web} já desenvolvida e implementada utilizando o \emph{Framework web Laravel}, a qual foi desenvolvida sem a utilização de testes automatizados. Utilizaram-se procedimentos e técnicas encontrados na literatura para identificação e planejamento de testes para criar os cenários de testes de integração necessários para testar casos de uso da aplicação \emph{web} de forma eficaz. Foi proposta uma representação do sistema usando mapas mentais que permite visualizar os testes necessários mais facilmente. Esses testes foram automatizados utilizando a ferramenta \emph{PHPUnit}, configurada por padrão em projetos que utilizam o \emph{framewok Laravel}. Os testes foram executados, e seus resultados  analisados. A partir da análise dos resultados e das práticas utilizadas no planejamento dos testes e implementação dos \emph{scripts} propôs-se um conjunto de recomendações para implementação de testes automatizados.

\textbf{Palavras-chave}: Testes Automatizados. Testes de Integração. Qualidade de Software. \emph{Laravel}. \emph{PHPUnit}.

\end{SingleSpacing}
\end{resumo}

