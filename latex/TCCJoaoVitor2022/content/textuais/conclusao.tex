% CONCLUSÃO--------------------------------------------------------------------

\chapter{CONSIDERAÇÕES FINAIS}
\label{chap:conclusao}

    O trabalho apresentou um estudo de caso da implementação de testes funcionais de integração automatizados em uma aplicação web, a partir do processo de planejamento dos testes, do processo de implementação dos \emph{scripts} e do conjunto de recomendações propostas e sua aplicação em um projeto real.
    
    Utilizando a documentação de casos de uso existente e a aplicação de técnicas de teste e o método de representação de cenários de teste utilizando diagramas baseados em mapas mentais, os testes foram planejados e documentados nos diagramas. Através da realização dessas atividades, percebeu-se os benefícios para a identificação dos cenários de teste que devem ser automatizados, além da representação dos mesmos em uma forma de documentação pouco extensa, mas que transmite as informações necessárias.
    
    Baseado nos diagramas produzidos e utilizando as informações por eles fornecidas, foi realizada a implementação dos \emph{scripts} de teste de integração automatizados. Os diagramas auxiliaram na construção dos \emph{scripts}, possibilitando identificar quais informações presentes no diagrama deveriam compor cada parte dos \emph{scripts}, além da quantidade de cenários ser a mesma da quantidade de métodos de teste criados.
    
    A execução dos testes automatizados mostrou-se rápida, levando poucos segundos para executar uma grande quantidade de testes e asserções, mostrando os testes automatizados como bons candidatos para serem utilizados em testes de regressão que executam os mesmos cenários de teste frequentemente.
    
    O conjunto de recomendações produzido mostrou-se benéfico, auxiliando nas atividades de implementação e execução dos testes automatizados, podendo ser aplicado em novos projetos.
    
    Algumas dificuldades para identificação de como realizar as asserções foram encontradas durante a implementação dos \emph{scripts}. Essas dificuldades foram ocasionadas principalmente pela forma que o sistema foi implementado sem considerar a testabilidade \footnote{característica que diz respeito a facilidade de se testar um software} do software durante o desenvolvimento.
    % incluir nota de rodapé sobre a testabilidade
    
    Portanto é possível concluir que a implementação de testes automatizados é um processo capaz de trazer benefícios para as atividades de teste de um projeto, através da execução rápida, possibilitando execuções frequentes. Entretanto é um processo que demanda conhecimentos técnicos de níveis, tipos e planejamento de testes, de codificação utilizando a linguagem de programação do software e das ferramentas de automação de testes adequadas para os tipos e níveis dos testes realizados. Além dos conhecimentos, devem ser empregadas as recomendações das documentações das ferramentas para automação de testes. Por fim, as recomendações propostas por esse trabalho têm potencial para auxiliar na realização desse processo.
    
    Como trabalhos futuros, existe a possibilidade de estender o processo realizado na criação de cenários de testes de integração automatizados para outros casos de uso do sistema. Além disso, é possível estender a aplicação dos testes automatizados para outros níveis e tipos de testes. Também é possível estudar as práticas que devem ser utilizadas e evitadas durante o desenvolvimento de um software visando uma melhor testabilidade e mesmo uma maior qualidade do software.
    

%O trabalho demonstrou que os processos de desenvolvimento de software tal como as atividades de teste de software passaram por muitas evoluções com o passar dos anos. Inicialmente as atividades de desenvolvimento e de teste de software eram realizadas por equipes distintas que não compartilhavam conhecimentos entre si. 

%Com o surgimento dos métodos ágeis as responsabilidades por desenvolvimento e teste passaram a ser do time de desenvolvimento. Além da mudança nas responsabilidades, o software passou a ser entregue frequentemente e de forma incremental, e o uso de testes automatizados passou a ser amplamente adotado por times ágeis.

%Entretanto, a implementação de testes automatizados demanda conhecimentos de desenvolvimento de software e de teste de software. Dessa forma as equipes que desenvolvem software em projetos que adotam métodos ágeis necessitam de conhecimentos não só de codificação de software, mas também de planejamento de testes e de implementação de testes automatizados. As atividades de teste de software envolvem a utilização de técnicas para identificação e planejamento de testes, além de abordarem diferentes níveis e tipos de teste com diferentes focos. Além disso existem diferentes ferramentas para implementação de testes automatizados focadas em diferentes níveis e tipos de teste, além de serem mais apropriadas para linguagens de programação específicas, demandando também conhecimento em codificação na linguagem de programação apropriada para a ferramenta de testes.
%

%Visando atender esse cenário 